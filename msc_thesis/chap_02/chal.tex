\section{Challenges}
\label{sec:chal}

The reinforcement learning environment poses a variety of obstacles 
that we need to address and potentially make trade-offs among them~\cite{dulac-arnold_challenges_2019, sutton_reinforcement_1998}.

\subsection{Exploration Exploitation Dilemma}

A RL agent is supposed to maximize rewards (exploitation of knowledge) by observing the environment (exploration of environment). 
This gives rise to the exploration-exploitation dilemma which is an inevitable trade-off. 
Exploration means taking a range of acts to benefit from the consequences, which typically results in low immediate rewards but high rewards for the future. 
Exploitation means taking action that has been learned, and typically results in high immediate rewards but low rewards in the future. 

\subsection{Generalization and Curse of Dimensionality}

A RL agent should also be able to generalize experiences to act on previously unseen situations. 
This issue arises when state space and action space is high-dimensional since experiencing all possibilities is impractical. 
However, this is solved by introducing function approximators. In Deep Reinforcement Learning, neural networks are used as function approximators. 

\subsection{Delayed Consequences}

A RL agent should be aware of the reason of reward or punishment. 
Sometimes, a wrong action may cause punishment later. 
For example, once a self-driving car enters a dead end way, it is not punished unless the way ends. 
Therefore, once an agent gets a reward or punishment, it should be able to discriminate whether reward is caused by instant or past actions. 

\subsection{Partial Observability}

Partial observability is the absence of all required observations to infer the instant state. 
For instance, a driver may not need to know engine temperature or rotational speed of gears. 
Although driver is able to drive in that case, s/he would not be able to drive well on traffic in the absence of rear view mirror or side mirrors. 
In real world, most systems are partially observable. 
This problem is usually tackled by incorporating observation history from the agents memory for action selection. 

\subsection{Safety of Agent}

Mechanical agents can kill or degrade themselves and their surroundings during the learning process. 
This safety problem is important on both exploration stage and full operation. 
In addition, it is difficult to analyze an RL agent's policy, that is, it is uncertain what the agent do in unseen situation. 
Simulation of environment is a good way to train the agent with safety, however, this causes an incomplete learning due to inaccuracy compared to the real environment. 
