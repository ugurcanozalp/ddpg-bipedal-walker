\chapter{BIPEDAL WALKING BY TWIN DELAYED DEEP DETERMINISTIC POLICY GRADIENTS}
\label{chap:exp_setup}

\section{Details of the Environment}

\textbf{Observation Space}: Hull angle, hull angular velocity, translational velocity on two dimension, joint positions, joint angular speeds, leg ground concats and 10 lidar rangefinder measurements. Details are summarized at Table \ref{table:bpw_obs_space}

\begin{table}[h!]
	\begin{center}
	\begin{tabular}{cccc}
		\textbf{Num} & \textbf{Observation} & \textbf{Max} & \textbf{Max} \\
		\hline
		0  & Hull Angle & $0$ & $2\pi$ \\
		1  & Hull Angular Vel & $-\infty$ & $+\infty$ \\
		2  & Vel x & $-1$ & $+1$ \\
		3  & Vel y & $-1$ & $+1$ \\
		4  & Hip 1 Joint Angle & $-\infty$ & $+\infty$ \\
		5  & Hip 1 Joint Speed & $-\infty$ & $+\infty$ \\
		6  & Knee 1 Joint Angle & $-\infty$ & $+\infty$ \\
		7  & Knee 1 Joint Speed & $-\infty$ & $+\infty$ \\
		8  & Leg 1 Ground Contact Flag & $0$ & $1$ \\
		9  & Hip 2 Joint Angle & $-\infty$ & $+\infty$ \\
		10  & Hip 2 Joint Speed & $-\infty$ & $+\infty$ \\
		11  & Knee 2 Joint Angle & $-\infty$ & $+\infty$ \\
		12  & Knee 1 Joint Speed & $-\infty$ & $+\infty$ \\
		13  & Leg 1 Ground Contact Flag & $0$ & $1$ \\
		14-23  & Lidar measures  & $-\infty$ & $+\infty$
	\end{tabular}
	\end{center}
	\caption{Observation Space of Bipedal Walker}
	\label{table:bpw_obs_space}
\end{table}

\textbf{Action Space}: Torque provided to knee and pelvis joints of both legs. Details are presented in Table \ref{table:bpw_act_space}.

\begin{table}[h!]
	\begin{center}
		\begin{tabular}{cccc}
			\textbf{Num} & \textbf{Observation} & \textbf{Max} & \textbf{Max} \\
			\hline
			0  & Hip 1 Torque & $-1$ & $+1$ \\
			1  & Hip 2 Torque & $-1$ & $+1$ \\
			2  & Knee 1 Torque & $-1$ & $+1$ \\
			3  & Knee 2 Torque & $-1$ & $+1$ \\
		\end{tabular}
	\end{center}
	\caption{Action Space of Bipedal Walker}
	\label{table:bpw_act_space}
\end{table}

\textbf{Rewarding}: Directly proportional to distance traveled forward, +300 points given if agent reaches end of path. -100 point if agent falls, and small amount of negative reward proportional to applied motor torque (preventing applying unnecessary torque).

\subsection{Partial Observability}
In DRL, partial observability is handled by 2 ways in literature \cite{dulac-arnold_challenges_2019}. First is incorprating fixed number of last observations while second way is updating hidden belief state using recurrent neural network at each time step. 

Our approach is using fixed number of past states into LSTM, BiLSTM and Transformer based networks. 

\section{RL Method and hyperparameters}
TD3 algorithm is used for learning task. For all networks following hyperparameters are used.

\section{Proposed Neural Networks}

For all networks, varing backbones used to encode state information from observations for both actor and critic networks. As backbones, following networks are proposed.

\begin{itemize}
	\item Feed Forward Network with residual connection (hidden dim: 128, feed forward dim: 384)
	\item LSTM (hidden dim: 128, number of layer: 1)
	\item Bidirectional LSTM (hidden dim: 96, number of layer: 1)
	\item Transformer (hidden dim: 128, feed forward dim: 256, number of layer: 1)
\end{itemize}

Actions are passed through feed forward layer with GELU activation and summed up by state encoding. Then, this summation is again passed through feed forward layer with GELU activation. Lastly, a linear layer is used for critic estimation in critic network and feed forward layer with tanh activation for action estimation in actor network.

\subsection{Feed Forward Network}

\subsection{Long Short Term Memory}

\subsection{Transformer (Pre-layer Normalized)}

\section{Results}