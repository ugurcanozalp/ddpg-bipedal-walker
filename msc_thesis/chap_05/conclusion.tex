\chapter{CONCLUSION AND FUTURE WORK}
\label{chap:conclusion_future_work}

\section{Conclusion}
\label{sec:conclusion}
In this thesis, bipedal robot walking is investigated by deep  reinforcement learning due to complex dynamics. 
For real world robot control by RL, simulation is an important step. 
Usually, models are pretrained in simulation environment before learning in reality due to safety and exploration reasons. 

TD3 algorthim is used since it is robust and well suited for continuous control. 
Also, environment is slightly modified by reward shaping, halving simulation frequency, cancelling terminal state information at time limit so that learning becomes easier.  
As stated in previous chapters, most of the real world environments are partially observable. 
In BipedalWalker-Hardcore, the environment is also partially observable since agent cannot observe behind of it lacks of acceleration sensors which is better to have for controlling mechanical systems. 
Therefore, we used Long Short Term Memory and Transformer Neural Networks to capture more information from past observations compared to Residual Feed Forward Neural Network (RFFNN) using single instant observation as input. 

RFFNN model performed well thanks to carefully selected hyperparameters and modifications on the environment. 
Since there is no major differences between sequential models (Transformer), the problem is mainly about rewards and exploration, not partial observability. 

Although LSTM is used commonly for partially observed problems, results are the worst compared to other two models, but keeps learning slowly. 
Without knowing exact reason, it is probably due to functional properties. 

Transformer model worked with the best performance among all models. 
It is surprising because it is not succesfully used in RL problems in general. 
In natural language processing, this type of attention models completely replaced recurrent models, and our results seems promising for this in RL domain. 

